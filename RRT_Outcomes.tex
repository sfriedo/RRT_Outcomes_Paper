
%% bare_conf_compsoc.tex
%% V1.4b
%% 2015/08/26
%% by Michael Shell
%% See:
%% http://www.michaelshell.org/
%% for current contact information.
%%
%% This is a skeleton file demonstrating the use of IEEEtran.cls
%% (requires IEEEtran.cls version 1.8b or later) with an IEEE Computer
%% Society conference paper.
%%
%% Support sites:
%% http://www.michaelshell.org/tex/ieeetran/
%% http://www.ctan.org/pkg/ieeetran
%% and
%% http://www.ieee.org/

%%*************************************************************************
%% Legal Notice:
%% This code is offered as-is without any warranty either expressed or
%% implied; without even the implied warranty of MERCHANTABILITY or
%% FITNESS FOR A PARTICULAR PURPOSE! 
%% User assumes all risk.
%% In no event shall the IEEE or any contributor to this code be liable for
%% any damages or losses, including, but not limited to, incidental,
%% consequential, or any other damages, resulting from the use or misuse
%% of any information contained here.
%%
%% All comments are the opinions of their respective authors and are not
%% necessarily endorsed by the IEEE.
%%
%% This work is distributed under the LaTeX Project Public License (LPPL)
%% ( http://www.latex-project.org/ ) version 1.3, and may be freely used,
%% distributed and modified. A copy of the LPPL, version 1.3, is included
%% in the base LaTeX documentation of all distributions of LaTeX released
%% 2003/12/01 or later.
%% Retain all contribution notices and credits.
%% ** Modified files should be clearly indicated as such, including  **
%% ** renaming them and changing author support contact information. **
%%*************************************************************************
\documentclass[conference,compsoc]{IEEEtran}

% *** CITATION PACKAGES ***
%
\ifCLASSOPTIONcompsoc
  % IEEE Computer Society needs nocompress option
  % requires cite.sty v4.0 or later (November 2003)
  \usepackage[nocompress]{cite}
\else
  % normal IEEE
  \usepackage{cite}
\fi



\begin{document}
\title{Prediction of Patient Outcomes after \\Renal Replacement Therapy (RRT) in the ICU }


% author names and affiliations
% use a multiple column layout for up to three different
% affiliations
\author{\IEEEauthorblockN{Harry Freitas Da Cruz}
\IEEEauthorblockA{Hasso Plattner Institute (HPI) \\
Enterprise Platform and Integration Concepts \\
Potsdam, Germany\\
Email: Harry.FreitasDaCruz@hpi.de}
\and
\IEEEauthorblockN{Siegfried Horschig}
\IEEEauthorblockA{Hasso Plattner Institute (HPI) \\
Enterprise Platform and Integration Concepts \\
Potsdam, Germany\\
Email: siegfried.horschig@student.hpi.de}
}
\maketitle

\begin{abstract}
In order to compensate impairments of the renal system in the human body, artificial methods in the form of renal replacement therapy (RRT), called dialysis, have to be introduced.
Many parameters of the dialysis can be adjusted and the outcome of the procedure may change with different patient characteristics. 
In this paper, we introduce a clinical decision support system to predict the effect of a given dialysis on a patient while in the intensive care unit (ICU). \\
For this purpose, we employ two kinds of machine learning models: Bayesian Rule Lists (BRL) and Deep Neural Networks (DNN). Although the DNN may provide better accuracy, its decision making is not easily interpretable for humans. For this reason, we use mimic learning as a method to make the DNN interpretable. \\
Results show us that the DNN outperforms our BRL classifier as expected, but by a rather small margin.
For the mimic learning process, we used a bayesian ridge regression model.
Even though the regression model performs worse when training as a mimic model as opposed to drectly on the data, it provides some insight into the inner workings of the DNN.
\end{abstract}


\section{Introduction}
The renal system in the human body has the purpose to eliminate wastes from the body and control levels of certain substances in the blood. 
If this system is impaired, for example due to Acute Kidney Injury (AKI), artificial methods in the form of Renal Replacement Therapy (RRT) have to be introduced, more commonly known under the term of dialysis. \\
There are different options for dialysis available.
One example is the hemodialysis, where the patient's blood is pumped through a dialyzer, inside of which is a liquid called dialysate.
This liquid's composition determines which substances should be filtered out of the blood.
The dialyzer separates the blood and the dialysate though a partially permeable membrane, allowing for the filtering of the blood through osmosis.
Another example for the dialysis is the peritoneal dialysis, which uses the peritoneal cavity inside the patient as a container for the dialysate. \\
The dialysis’ outcomes are highly dependent on both the patient’s characteristics and the parameters as well as the type of the dialysis. 
So, usually, patients undergoing the peritoneal dialysis experience lesser health issues related to the dialysis than those undergoing hemodialysis, as there is less pressure on the circulatory system.
On the other hand, the hemodialysis is more efficient in such a way that it needs less time for the same amount of filtration. \\
Especially the hemodialysis is a costly process which needs specialized equipment and therefore has many parameters to be tuned. 
These include, but are not limited to the duration of the process, the filtration rate and flow rates of the blood and dialysate.
The goal of this paper is to make a prediction of the patient outcome
To support clinical decision making, our goal is to develop a Clinical Prediction Model (CPM), which predicts patient outcomes based on their data while in the Intensive Care Unit (ICU). Additionally, we want the resulting model to be interpretable. Especially in the medical context, it is important to know why a specific decision was made to ensure patient safety and validate that decision. More powerful models, such as Deep Neural Networks, do not expose their decision process in a human-readable fashion and are thus non-interpretable. Our goal was also to make such a non-interpretable model interpretable.





\section{Related Work}
Relating to and building on top of existing models\cite{dialysisLength} and studies \cite{barrett1997prediction} \cite{Schwenger2012}, we aim to develop a \emph{Clinical Prediction Model}\cite{cpmLee}. 

\section{Conclusion}
The conclusion goes here.


\bibliographystyle{IEEEtran}
\bibliography{references}

\end{document}


